%注意保存的文件名不能有中文
%编译方式:XeLaTeX    
        
\documentclass[12pt,a4paper]{article}  
	\usepackage{xeCJK}	%中文环境
		\usepackage{float}	%支持浮动体参数H(大写)
%注意右下角换成utf-8的
\usepackage{graphicx}	%插图命令包
    \setCJKmainfont{SimSun}	%设置字体
\usepackage{titlesec}	
\usepackage{fontspec}
\usepackage{geometry}
\usepackage{listings}
\geometry{left=2.5cm,right=2.5cm,top=2.5cm,bottom=2.5cm}
\newfontfamily\arial{Arial}
\newcommand{\sectionfontsize}{\fontsize{12pt}{12pt}\selectfont}
%\titleformat{\section}[hang]{\bfseries\arial\sectionfontsize}{\thesection}{1em}{}{}


\title{第一章:了解Python}	%题目作者日期
\author{田雅夫}
\date{\today}
\bibliographystyle{plain}	%参考文献格式


\begin{document}

\maketitle
\begin{figure}[P]
\centering
\tableofcontents
\end{figure}

\section{Python与C的差异}
\begin{enumerate}
\item 编译器,解释器的差异
\item 抽象层次的差异
\item 语法的差异%更优雅,可读性更强 
\item 执行效率的差异
\end{enumerate}

\subsection{编译器,解释器的差异}
在C语言中,代码是经过整体编译变成机器码然后执行的.而在Python中代码是按行执行的.

\section{解释器语言的特点}

\section{弱类型语言注意事项}

\section{良好的代码风格}


测试用文档模板
    \end{document}  









































