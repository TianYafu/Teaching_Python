\documentclass{beamer}  
\usetheme{CambridgeUS}  
\usepackage[UTF8,noindent]{ctexcap}
\usepackage{fontspec}  
%\setsansfont{楷体} % font name is case-sensitive  
       
\begin{document}  
\title{Python入门训练}  
\subtitle{[0].Python基本操作}
\author{田雅夫}
\institute{控制模型创新实验室}
\date{\today}
\frame{\titlepage}  
    \begin{frame} {教学计划}{知识,技能和训练}

    \begin{itemize}  
    \item Python知识的学习  
    \item 良好的习惯与代码风格
    \item 足够的熟练程度 
	\end{itemize}  
	\end{frame}  
\section{Python知识的学习}


\begin{frame}{为什么是Python}
\begin{enumerate}
\item 脚本式语言
\item 快速开发
\item 简单,容易上手
\item 大量好用的外部库,详细的文档,例程
\item 易于调试,维护
\item 发展前景良好

\end{enumerate}

\end{frame}


\subsection{知识性目标}
\begin{frame}{知识性目标}
\begin{enumerate}
\item 了解Python
\item Python基本开发环境的构建
\item 基础语法,操作.
\item 列表推导和简单的 $\lambda$演算
\item 简单的GUI构建
\item 串口库的使用
\item 字符串,元组和字典
\item OOP编程,类,对象和方法.
\item 文件操作
\item 调试的技巧
\item 简单实用的标准库操作
\item 建立自己的工程
\end{enumerate}
\end{frame}

\subsection{能力性目标}
\begin{frame}{能力性目标}
\begin{enumerate}
\item 良好代码缩进的能力
\item 良好的命名风格
\item 良好的注释习惯
\item 代码抽象,封装的能力
\item 根据需求编写代码的能力
\item 代码调试,改正的能力

\end{enumerate}
\end{frame}

\subsection{工程目标}

\begin{frame}{工程目标}
\begin{enumerate}
\item 用Python实现串口通信
\item 用Python作为上位机控制STM32
\item 基于STM32控制流水灯,电机,舵机\dots
\item 用GUI封装控制程序
\item 编写产品说明,文档,帮助文件
\end{enumerate}
\end{frame}

\section{学习方法}
\begin{frame}
我知道刚才讲的东西有很多你们都不明白,例如串口,例如GUI,例如元组,字典,OOP\dots\\

程序设计不是单单凭借听课就能够理解的,必须要写代码,写大量的代码.用的多了自然就明白了.
\end{frame}

\begin{frame}{学习方法}
\begin{itemize}
\item 每次讲课我只讲重要的,容易出错的部分
\item 用大量的时间来编写各种各样的程序
\item 编写程序的过程中可以参考任何资料,但必须独立完成.
\item 上台讲解你所编写的程序

\end{itemize}
\end{frame}
\section{基础开发环境}
\section{Python开发环境的构建}
\begin{frame}{安装windows下32位Python}
\begin{enumerate}
\item 双击安装
\item 添加环境变量
\item 在命令行下启动Python
\item 启动IDLE
\item 建立一个存放代码的文件夹
\item 保护眼睛的字体和配色方案
\item 动手吧\dots
\end{enumerate}


\end{frame}

\section{交互模式和脚本模式}

\begin{frame}{交互模式}
\begin{itemize}
\item 方便的调试模块
\item 程序输出的地方
\item 写错了怎么办?
\item 完全没办法保存啊

\end{itemize}
\end{frame}

\begin{frame}{脚本模式}
\begin{itemize}
\item 可以重用的脚本
\item 重用以前的脚本
\item 保存的时候注意事项
\item 运行你的脚本

\end{itemize}

\end{frame}

\section{基本操作}

\begin{frame}{变量}

\begin{enumerate}
\item 使用变量前完全不用定义类型
\item 但是要声明以及赋初值
\item 不定义类型也不见得就是好事,指不定什么时候类型就悄悄地改变了呐\dots
\item 有些地方因为用的太频繁所以声明和初值都省略了...
\item 解释器语言变量的作用域
\item 变量名称变颜色的话,最好换一个.
\item 不要用两个下划线开头
\item type语句十分有用

\end{enumerate}
\end{frame}

\begin{frame}{格式输入与输出}
\begin{enumerate}
\item print,没有f
\item 用逗号的话可以不换行
\item C语言的格式输出还是有效的
\item 输入语句更简洁了
\item 练习,大量的练习

\end{enumerate}
\end{frame}


\begin{frame}{操作符与表达式}
\begin{enumerate}
\item 加减乘除取余都没什么变化,除法还要注意类型
\item 逻辑与和或变得更清楚了
\item 字符串不但能做加法,还能做乘法
\item 先乘除后加减,剩下的东西都加括号
\item 自增运算符(++,-- --)不能用了,但简化运算符(+=,-=)还是可以用的

\end{enumerate}
\end{frame}

\begin{frame}{条件语句if}
\begin{enumerate}
\item 三目运算符什么的就不要指望了
\item if语句和C语言的差不多
\item 冒号问题
\item 缩进!一定要注意缩进!

\end{enumerate}
\end{frame}


\begin{frame}{定义函数}
\begin{enumerate}
\item 主函数什么的完全没有(或者说仅仅是看不到了),程序的入口就是第一行
\item 用def定义函数
\item 声明什么的就不要想了,不过可以单独写一个文件然后import进去
\item 良好的抽象封装习惯可以让你少写很多代码,看代码的人也会很开心
\item 刀叉的作用是将食物分解方便食用,函数也一样(分治法)

\end{enumerate}
\end{frame}

\begin{frame}{while循环}
\begin{enumerate}
\item 和for的用法差不多,学过C语言的人一看就明白了
\item 缩进,还是缩进!
\end{enumerate}
\end{frame}

\begin{frame}{列表(list)}
\begin{enumerate}
\item C语言中列表是定长的,而列表是可变长度的
\item 可以把列表当成单链表来用
\item 下标是负数会怎么样?
\item 遍历一个列表
\item 列表操作 (乘法尤其要注意!)
\item 列表切片
\item 求列表长度

\end{enumerate}
\end{frame}

\begin{frame}{列表方法}
\begin{enumerate}
\item 枚举(enumerate)
\item 附加(append)
\item 删除(pop,del,remove的区别) %pop有返回值,del没有,remove是根据元素来删除
\item 列表转字符串(list,join)
\item 字符串转列表(split)
\item 统计出现次数(count)
\end{enumerate}
\end{frame}



\begin{frame}{$\lambda$表达式与函数式编程}
\begin{enumerate}
\item 匿名函数:lambda
\item 迭代更新的简单方法:map
\item 把列表缩减为单值:reduce
\item 过滤函数:filter

\end{enumerate}
\end{frame}



\begin{frame}
\begin{enumerate}
\item

\end{enumerate}

\end{frame}


\end{document}  



























