%注意保存的文件名不能有中文
%编译方式:XeLaTeX    
        
\documentclass[12pt,a4paper]{article}  
	\usepackage{xeCJK}	%中文环境
		\usepackage{float}	%支持浮动体参数H(大写)
%注意右下角换成utf-8的
\usepackage{graphicx}	%插图命令包
    \setCJKmainfont{SimSun}	%设置字体
\usepackage{titlesec}	
\usepackage{fontspec}
\usepackage{geometry}
\usepackage{listings}
\geometry{left=2.5cm,right=2.5cm,top=2.5cm,bottom=2.5cm}
\newfontfamily\arial{Arial}
\newcommand{\sectionfontsize}{\fontsize{12pt}{12pt}\selectfont}
%\titleformat{\section}[hang]{\bfseries\arial\sectionfontsize}{\thesection}{1em}{}{}


\title{需求文档(2)}	%题目作者日期
\author{田雅夫}
\date{\today}
\bibliographystyle{plain}	%参考文献格式


\begin{document}
\maketitle

\begin{figure}[H]
1.给定一个正整数m,实现m的阶乘

\end{figure}

\begin{figure}[H]
2,给定一个正整数m,找出比m小的所有质数

\end{figure}

\begin{figure}[H]
3.已知一个网站图片的URL格式如下:
\begin{quote}
$url:http://www.python.org/s/pic/***/*******$
\end{quote}

其中***部分取值为000至233,*******部分为a00001到a00005,b00001到b00005\dots,e00001到e00005
\begin{enumerate}
\item 建立两个列表,第一个列表中元素为[000,233]中正整数,第二个列表中为a00001到a00005,b00001到b00005\dots,e00001到e00005中所有元素.
\item 格式输出所有的图片URL
\end{enumerate}
\end{figure}

\begin{figure}[H]
4.用有序单链表表示集合,实现集合的交、并和差运算。
 对集合中的元素,用有序单链表进行存储。
\end{figure}

\begin{figure}[H]
5.设有编号为1,2,…,n的n(n>0)个人围成一个圈,每个人持有一个密码m。从第一个人开始报数,报到m时停止报数,报m的人出圈,再从他的下一个人起重新报数,报到m时停止报数,报m的出圈,……,如此下去,直到所有人全部出圈为止。当任意给定n和m后,设计算法求n个人出圈的次序。

\end{figure}

\begin{figure}[H]
6.实现栈,队列,循环列表三种数据结构:

要求:定义函数实现入栈,出栈,如队列,出队列,自行决定演示方式.

\end{figure}

\begin{figure}[H]
7.二叉树的一个节点可以看做一个列表[节点编号,节点值,左孩子,右孩子],这样的话一个二叉树可以变成一个二维列表.试着生成如下的一棵二叉树.
\includegraphics[scale=1.0]{bitree.pdf}

\end{figure}

\begin{figure}[H]
8.下面生成二维数组的代码是错误的(它确实可以生成二维数组但它是错误的),请分析它错误在哪里.改正代码使其正确.
\begin{lstlisting}[language=Python]
a = []
def matrix(n):
    vector = [0]*n
    for i in range(n):
        a.append(vector)
\end{lstlisting}
\end{figure}

\begin{figure}[H]
9.冒泡排序,插入排序或是任意一种排序算法排序以下列表:

[1,7,3,9,2,4,8,2,5,7,4,3]

\end{figure}

\begin{figure}[H]
10.验证码解析装置: 

有如下的验证码:

A23D-36CA-84SE-57EW

该验证码以一整个字符串形式输入,要求
\begin{enumerate}
\item 1.将该验证码以列表形式储存,列表形式如下["A23D","36CA","84BC","57DA"]
\item 2.A对应:"14",B对应"23",C对应"8",D对应"25",将字符串翻译成["142325","36814","84238","572514"]
\item 3.将这四个字符串用int(a)语句转换成整型,并将其值相加,得到加和后的结果.

\end{enumerate}

\end{figure}


    \end{document}  









































