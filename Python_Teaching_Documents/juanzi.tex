%注意保存的文件名不能有中文
%编译方式:XeLaTeX    
        
\documentclass[12pt,a4paper]{article}  
	\usepackage{xeCJK}	%中文环境
		\usepackage{float}	%支持浮动体参数H(大写)
%注意右下角换成utf-8的
\usepackage{graphicx}	%插图命令包
    \setCJKmainfont{SimSun}	%设置字体
\usepackage{titlesec}	
\usepackage{fontspec}
\usepackage{geometry}
\usepackage{listings}
\geometry{left=2.5cm,right=2.5cm,top=2.5cm,bottom=2.5cm}
\newfontfamily\arial{Arial}
\newcommand{\sectionfontsize}{\fontsize{12pt}{12pt}\selectfont}
%\titleformat{\section}[hang]{\bfseries\arial\sectionfontsize}{\thesection}{1em}{}{}


\title{控制模型创新实验室暑期培训预测试}	%题目作者日期
\author{姓名:\phantom{5cm}\\学号:\phantom{5cm}}
\date{\today}
\bibliographystyle{plain}	%参考文献格式


\begin{document}

\maketitle



\section{能力与态度测试}

1.GitHub,sorceforge,博客园,codeproject,codeplex,csdn,pudn,stackoverflow,知乎,Quora,你熟悉几个?都是干什么的?



2.你认为机器人是干什么用的?设计一个你理想的机器人功能.



3.你最向往的技术,领域,方向是什么?你对它有什么了解?


4.IT行业的四大俗是什么?近两年你听过最多的高新技术词汇是什么?



5.如果后面两部分的题你一道都不会做,怎么办?


\section{电子电路能力测试}

\subsection{}
读下面两个电路图,思考这两个电路分别具有什么样的功能。

\begin{figure}[H]
\includegraphics[scale = 0.8]{circuit01.jpg}
\qquad
\includegraphics[scale=0.65]{circuit03.jpg}
\end{figure}


\subsection{}
(1)三极管的作用是放大电流,其输入特性(伏安特性)如图,思考该电路中为什么出现了两个三极管,该电路的作用是什么?该电路可能存在什么问题?

(2)对比上一道题图(1)中的两个二极管与本题图中的两个三极管,指出二者在作用上的差别。
\begin{figure}[H]
\includegraphics[scale = 0.9]{circuit02.pdf}

\end{figure}

\begin{figure}[H]
\includegraphics[scale = 0.8]{circuit04.jpg}

\end{figure}


\section{程序设计能力测试}
1.m个人围成一圈,从第一个人开始顺序报号1,2,3\dots .凡报到n者退出圈子(输入正整数m,n).编写代码给出出圈顺序.


    \end{document}  









































