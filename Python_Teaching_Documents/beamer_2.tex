\documentclass{beamer}  
\usetheme{Singapore}  
\usepackage[UTF8,noindent]{ctexcap}
\usepackage{fontspec}  
%\setsansfont{楷体} % font name is case-sensitive  
       
  
\title{Python入门训练}  
\subtitle{第二章:面向对象编程}
\author{田雅夫}
\institute{控制模型创新实验室}
\date{\today}

     
\begin{document}  


\frame{\titlepage} 
 
\section{什么是OOP}
\begin{frame}{面向对象编程}
\begin{enumerate}
\item 类
\item 成员变量
\item 类与对象与实例化
\item 方法
\item 定义一个类
\end{enumerate}
\end{frame}  

\section{基本的类操作方法}
\begin{frame}{Python的类操作方法}
\begin{enumerate}
\item 定义类的时候一定不要忘了的两个方法
\item 类的实例化
\item 用列表存储各个实例
\item 删除一个实例
\item 删除一个类
\end{enumerate}
\end{frame}

\begin{frame}{在Python中一切皆对象}
\begin{enumerate}
\item 列表对象
\item 字符串对象
\item 函数对象
\item enumerate对象
\item 查看一个对象(getattr,dir())

\end{enumerate}

\end{frame}

\begin{frame}{类的继承}
\begin{enumerate}
\item 基类,子类
\item 类方法的传递
\item \_\_init\_\_方法的覆盖 
\end{enumerate}

\end{frame}

\begin{frame}{操作符重载}
\begin{enumerate}
\item call
\item getitem
\item 双目运算符的重载add,mul,and
\item gt,ge缺一个的话会发生很有意思的事情
\item 重载的len也只能输出int类型

\end{enumerate}

\end{frame}

\end{document}  






